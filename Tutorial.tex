\documentclass{article}
\usepackage[ngerman]{babel}
\usepackage[utf8]{inputenc}

\title{\LaTeX\ Dokumentation}
\author{Thomas Davies}
\setlength{\parindent}{0cm}


\begin{document}
	\maketitle
	\section{Aufbau eines \LaTeX-Dokuments:}
	\begin{verbatim}
	\documentclass{article}
	\usepackage[ngerman]{babel} % Lädt das Sprachpaket
	\usepackage[utf8]{inputenc} % Lädt die utf8 Kodierung
	
	\title{\LaTeX\ Dokumentation} % Setzt den Titel des Dokuments
	\author{Thomas Davies} % Setzt den Urheber des Dokuments
	
	\setlength{\parindent}{0cm} % Vermeidet, dass Paragraphen eingerückt werden
	
	\begin{document} % Umgebung des Dokuments
	\maketitle % Erzeugt eine Teilüberschrift
		
	\end{document} % Beendet die Dokumentumgebung
	\end{verbatim}
	
	\section{Gliederung eines \LaTeX~Dokuments}
	Eine Gliederung ersten Grades erreicht man mit:\\
	\verb|\section{Gliederung ersten Grades}|
	
	Eine Gliederung zweiten Grades erreicht man mit:\\
	\verb|\subsection{Gliederung zweiten Grades}|
	
	Eine Gliederung dritten Grades erreicht man mit:\\
	\verb|\subsubsection{Gliederung dritten Grades}|
	
	Eine Gliederung vierten Grades erreicht man mit:\\
	\verb|\paragraph{Gliederung vierten Grades}|
	
	Eine Überschrift ersten Grades aber ohne Nummerierung erhält man mit:\\¸
	\verb|\section*{Überschrift ohne Nummerierung mit*}|
	
\end{document}



